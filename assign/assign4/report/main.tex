\documentclass{article} % For LaTeX2e
\usepackage{nips15submit_e,times}
\usepackage{hyperref}
\usepackage{url}
\usepackage{graphicx}
\usepackage{amsfonts}
\usepackage{amssymb}
\usepackage{float}
\usepackage{listings}
\usepackage{xcolor}
\usepackage{amsmath}
%\documentstyle[nips14submit_09,times,art10]{article} % For LaTeX 2.09


\title{Problem Set 4 for Machine Learning 15 Fall}


\author{
Jingyuan Liu\\
AndrewId: jingyual\\
\texttt{jingyual@andrew.cmu.edu} \\
}


\newcommand{\fix}{\marginpar{FIX}}
\newcommand{\new}{\marginpar{NEW}}
\newcommand{\argmin}{\arg\!\min}
\newcommand{\norm}[1]{\left\lVert #1 \right\rVert}
\newcommand{\abs}[1]{\left\lvert #1 \right\rvert}
\newcommand{\inner}[1]{\left\langle #1 \right\rangle}


\nipsfinalcopy % Uncomment for camera-ready version


\begin{document}
\maketitle



\section{Neural Networks}


For convenience, we can denote the nodes in the first layer as $n_1, n_2$, the
second layer $n_3$.

\subsection{Neural network for regression}
\textbf{1.1.1. Simulating linear regression}

In this scenario, every node should use L.

\begin{equation}
n_1 = c(x_1 w_1 + x_2 w_3), \quad
n_2 = c(x_1 w_2 + x_2 w_4)
\end{equation}

\begin{equation}
y = c n_3 =  c(n_1 w_5 + n_2 w_6)
\end{equation}

\begin{equation}
\qquad \qquad \qquad \qquad \quad
= c^2 ((x_1 w_1 + x_2 w_3) w_5 + (x_1 w_2 + x_2 w_4) w_6)
\end{equation}

\begin{equation}
\qquad \qquad \qquad \qquad \quad
= c^2 (w_1 w_5 + w_2 w_6) x_1 + c^2 (w_3 w_5 + w_4 w_6) x_2
\end{equation}

\begin{equation}
\beta_1 = c^2 (w_1 w_5 + w_2 w_6),  \quad
\beta_2 = c^2 (w_3 w_5 + w_4 w_6)
\end{equation}

\textbf{1.1.2. Derive $\beta_1$ and $\beta_2$}

In this scenario, $n_1$ and $n_2$ should use L, and $n_3$ should use S.

\begin{equation}
n_1 = c(x_1 w_1 + x_2 w_3), \quad
n_2 = c(x_1 w_2 + x_2 w_4)
\end{equation}

\begin{equation}
p(n_3 = 1 \mid X) = \frac{1}{1 + exp(-(n_1 w_5 + n_2 w_6))}
\end{equation}

\begin{equation}
\qquad \qquad \quad = \frac{exp(n_1 w_5 + n_2 w_6)}{1 + exp(n_1 w_5 + n_2 w_6)}
\end{equation}

According to the defination of S, we know that the Y = 1 when $n_3$ = 1, and Y = -1
when $n_3$ = 0. Therefore, we can derive that:

\begin{equation}
P(Y=1 \mid X) = p(n_3 = 1 \mid X) = \frac{exp(n_1 w_5 + n_2 w_6)}
{1 + exp(n_1 w_5 + n_2 w_6)}
\end{equation}

\begin{equation}
\beta_1 = c (w_1 w_5 + w_2 w_6),  \quad
\beta_2 = c (w_3 w_5 + w_4 w_6)
\end{equation}

\textbf{1.1.3. Derive $\alpha_1$ and $\alpha_2$}

In this scenario, $n_1$, $n_2$, and $n_3$ all should use S. For f1
and f2, they employ the same distribution as 1.1.2, so:

\begin{equation}
p(Y_1=1 \mid X, f1) = p (n_1=1 \mid x)
= \frac{exp(w_1 x_1 + w_3 x_2)}{1 + exp(w_1 x_1 + w_3 x_2)}
\end{equation}

\begin{equation}
p(Y_2=1 \mid X, f2) = p (n_2=1 \mid x)
= \frac{exp(w_2 x_1 + w_4 x_2)}{1 + exp(w_2 x_1 + w_4 x_2)}
\end{equation}

For $n_3$ and Y, we have:

\begin{equation}
n_3 = \frac{1}{1 + exp(n_1 w_5 + n_2 w_6)}, \quad
y = sign(\alpha_1 Y_1 + \alpha_2 Y_2)
\end{equation}

\begin{equation}
\alpha_1 = w_5, \quad
\alpha_2 = w_6
\end{equation}


\subsection{Convolutional Neural Networks}

\textbf{1.2.1 Parameters}

For the convolution layers, we have:

6 * (25 + 1) + (6 * 25 + 1) * 16 = 2572

For the fully connected layers, we have:

(1 + 16 * 5 * 5) * 120 + (120 + 1) * 84 + (84 + 1) * 10 = 59134

\textbf{1.2.2 Derive the gradients}

The gradient is:

\begin{equation}
\frac{\partial L}{\partial w_i^{(q)}} = 
\sum_{i=1}^{J^{(q)}} \frac{\partial L}{\partial z_j^{(q)}}
\frac{\partial z_j^{(q)}}{\partial w_i^{(q)}}
\end{equation}

\begin{equation}
\qquad =
\sum_{i=1}^{J^{(q)}} \frac{\partial L}{\partial z_j^{(q)}}
\sigma' (\sum_i w_i^{(q)} x_{ji}^{(q)}) x_{ji}^{(q)}
\end{equation}

Here $\sigma$ is the activation function.


\subsection{Gradient vanishing/explosion}

\textbf{1.3.1 Derive b1, the first layer bias}

We can derive the derivative of L w.r.t. b1 using the chain rule:

\begin{equation}
\frac{\partial L}{\partial b_1} =
\frac{\partial L}{\partial z_m} \frac{\partial z_m}{\partial z_{m-1}} ...
\frac{\partial z_{2}}{\partial z_{1}} \frac{\partial z_{1}}{\partial b_1}
\end{equation}

\begin{equation}
\qquad \qquad \qquad
= \frac{\partial L}{\partial z_m}
\cdot \sigma' (w_1 x + b_1)
\cdot \prod_{k=1}^m \frac{\partial z_{k+1}}{\partial z_k}
\end{equation}

\newpage
\textbf{1.3.2. (a) explain vanish trend}

As the activation function is given, we can derive:

\begin{equation}
\frac{\partial z_{k+1}}{\partial z_k} =
\frac{\partial \sigma (w_k z_k + b_k)}{\partial (w_k z_k + b_k)}
\cdot \frac{\partial (w_k z_k + b_k)}{\partial z_k}
\end{equation}

\begin{equation}
\qquad \qquad \qquad =
\sigma (w_k z_k + b_k) (1- \sigma (w_k z_k + b_k))
\cdot w_k
\end{equation}

As we know, $\sigma (x)$ and $1-\sigma (x)$ is smaller than 1, and $\abs{w_k}$
is smaller than 1. Therefore, according what we derived from 1.3.1, we would
konw that the $\frac{\partial L}{\partial b_1}$ tends to vanish when m is large.

\textbf{1.3.2. (b) explain vanish trend given large $\abs{w}$}

Given the form of the $\frac{\partial z_{k+1}}{\partial z_k}$, we can simple
consider this function:

\begin{equation}
f_{temp} = \frac{x \cdot exp(x)}{ (1+exp(x))^2}
\end{equation}

which is the same ``form'' of the derivative. As we can see, with increase of x,
the $f_{temp}$ would become smaller and would be smaller than 1. The trend of
$\frac{\partial z_{k+1}}{\partial z_k}$ is the same. When the $\abs{w}$ is
large, the total $\frac{\partial z_{k+1}}{\partial z_k}$ would still be smaller
than 1, because the decreasing trend of
$\sigma (w_k z_k + b_k) (1- \sigma (w_k z_k + b_k))$ would be more influential
than the increasing trend of $w_k$.

\textbf{1.3.3. Explain the ReL}

We could find that when $x > 0$:

\begin{equation}
\frac{\partial \sigma (x)}{\partial x} = 1
\end{equation}

Then in this case, we know that the parameters would vanish as the sigmoid
function.

\textbf{1.3.4. Prove the equation}

From the equation, we can derive:

\begin{equation}
\frac{\partial log p(v)}{\partial \theta_i} =
\frac{\partial log \sum_h p(v, h)}{\partial \theta_i}
\end{equation}

\begin{equation}
log \sum_h p(v,h) = log \sum_h exp(\sum_i \theta_i \phi_i (v,h))
- log (\sum_{v,h} exp (\sum_i \theta_i \phi_i (v,h)))
\end{equation}

For simplicity, we can use $\phi_i$ to replace $\phi_i (v,h)$,
the first part and second part:

\begin{equation}
\frac{\partial log \sum_h p(v,h)}{\partial \theta_i} =
\frac{1}{\sum_h exp(\sum_i \theta_i \phi_i)} \cdot
\frac{\partial \sum_h exp(\sum_i \theta_i \phi_i)}{\partial \theta_i} \cdot
\end{equation}

\begin{equation}
\qquad \qquad \qquad \quad =
\frac{1}{\sum_h exp(\sum_i \theta_i \phi_i)} \cdot
\sum_h exp(\sum_i \theta_i \phi_i) \cdot \phi_i
\end{equation}

\begin{equation}
= \sum_h \phi_i \frac{exp(\sum_i \theta_i \phi_i)}
{\sum_h exp(\sum_i \theta_i \phi_i)}
\end{equation}

Using bayes rule, we can derive the first part as:

\begin{equation}
p (h \mid v) = \frac{p(v,h)}{p(v)} = \frac{p(v,h)}{\sum_h p(v,h)}
\end{equation}

\begin{equation}
\qquad = 
\frac{exp(\sum_i \theta_i \phi_i)}{\sum_h exp(\sum_i \theta_i \phi_i)}
\end{equation}

\begin{equation}
\frac{\partial log \sum_h p(v,h)}{\partial \theta_i} =
\sum_h \phi(v,h) p (h \mid v)
\end{equation}

We can similarily derive the second part:

\begin{equation}
\frac{\partial \sum_{v,h} exp(\sum_i \theta_i \phi_i)}{\partial \theta_i}
= \frac{1}{Z} \cdot \frac{\partial \sum_{v,h} exp(\sum_i \theta_i \phi_i)}
{\partial \theta_i}
\end{equation}

\begin{equation}
\qquad \qquad \qquad \quad = 
\sum_{v,h} \phi_i \frac{exp(\sum_i \theta_i \phi_i)}{Z} 
\end{equation}

\begin{equation}
\qquad \qquad = 
\sum_{v,h} \phi_i p(v,h)
\end{equation}

Combine the two parts, we can get the conclusion:

\begin{equation}
\frac{\partial log p(v)}{\partial \theta_i} = \sum_h \phi_i (v,h) p(h \mid v)
- \sum_{v,h} \phi_i (v,h) p(v,h)
\end{equation}



\section{Support Vector Machines}


\subsection{Support Vector Regression}

\textbf{2.1.1. write the dual problem}

Suppose we have $\xi_i$ ofr each $x_i$:

\begin{equation}
\abs{f(x) - y} < \epsilon
\end{equation}

\begin{equation}
- \epsilon  < w^T x_i - y_i < \epsilon
\end{equation}

Let's set:

\begin{equation}
\abs{w^T x_i - y_i} - \epsilon = \xi_i
\end{equation}

\begin{equation}
L = \frac{1}{2} \norm{w}_2^2 + C \sum_{i=1} \xi_i 
- \sum_i \alpha_i (\epsilon + \xi_i - y_i + w x_i)
- \sum_i \alpha_i^{\star} (\epsilon + \xi_i +y_i - w x_i)
- \sum_i \beta_i \xi_i
\end{equation}

\begin{equation}
\alpha_i >= 0, \alpha_i^{\star} >= 0, \beta_i >= 0
\end{equation}

We have to satisfy saddle constraints condition, so

\begin{equation}
\partial_w L = w - \sum_i (\alpha_i - \alpha_i^{\star}) x_i = 0
\end{equation}

\begin{equation}
\partial_{\xi_i} L = C - \alpha_i -\alpha_i^{\star} - \beta_i = 0
\end{equation}

Take the the conditions back and we can get the form of dual problem of SVR:

\begin{equation}
maximize: - \frac{1}{2} \sum_{i,j=1} (\alpha_i - \alpha_i^{\star})
(\alpha_j - \alpha_j^{\star}) x_i^T x_j
- \epsilon \sum_i (\alpha_i + \alpha_i^{\star})
+ \sum_i y_i (\alpha_i - \alpha_i^{\star})
\end{equation}

\begin{equation}
subject \quad to: \sum_i (\alpha_i - \alpha_i^{\star}) = 0, \qquad
\alpha_i, \alpha_i^{\star} \in [ 0, C]
\end{equation}


\textbf{2.1.2. KKT Conditions}

The KKT condition for this problem is:

\begin{equation}
\partial_w L = w - \sum_i (\alpha_i - \alpha_i^{\star}) x_i = 0
\end{equation}

\begin{equation}
\partial_{\xi_i} L = C - \alpha_i -\alpha_i^{\star} - \beta_i = 0
\end{equation}

\begin{equation}
\epsilon + \xi_i - y_i + w^T x_i >= 0
\end{equation}

\begin{equation}
\epsilon + \xi_i^{\star} + y_i - w^T x_i >= 0
\end{equation}

\begin{equation}
\alpha_i >= 0, \alpha_i^{\star} >= 0, \beta_i >=0
\end{equation}

\begin{equation}
\alpha_i (\epsilon + \xi_i - y_i + w^T x_i) = 0
\end{equation}

\begin{equation}
\alpha_i^{\star} (\epsilon + \xi_i^{\star} + y_i - w^T x_i) = 0
\end{equation}

\textbf{2.1.3. Kernelized SVR}

The prediction rule is:

\begin{equation}
w = \sum_i (\alpha_i - \alpha_i^{\star}) x_i
\end{equation}

\begin{equation}
f(x) = \sum_i (\alpha_i - \alpha_i^{\star}) x_i^T x
\end{equation}

Add we can use kernel to transfer the x, therefore:

\begin{equation}
f(x_j) = \sum_i (\alpha_i - \alpha_i^{\star}) K(x_i, x_j)
\end{equation}

\textbf{2.1.4. Why using SVR primal}


We are using the primal of SVM and the dual of SVR, this is because we want to
compose the term:

\begin{equation}
\sum_i x_i x
\end{equation}

Therefore, we could use the kernel trick to implicitly transfer the feature of
the data to other spaces.


\subsection{Support Kernel Machines}

\textbf{2.2.1.(a) The minimum of the Lagrangian function}

Given L, the Lagrangian function, we could derive to optimize the w:

\begin{equation}
\frac{\partial L}{\partial w_j} = \frac{1}{2} \frac{\partial (\sum_j d_j
\norm{w_i}_2)^2}{\partial w_j} - \frac{\partial \alpha_i y_i \sum_j w_j^T
x_{ji}}{\partial w_j}
\end{equation}

We can have $\frac{\partial L}{\partial w_j} = 0$. Then we can both times the
$w_j^T$:

\begin{equation}
\frac{w_j d_j \sum_j \norm{w_j}_2 d_j }{\norm{w_j}_2} = \sum_i \alpha_i y_i x_{ji}
\end{equation}

\begin{equation}
\norm{w_j}_2 d_j \gamma = w_j^T \sum_i \alpha_i y_i x_{ji}
\end{equation}

\textbf{2.2.1.(b) Show that $\norm{\sum_i \alpha_i y_i x_{ji}}_2 < d_j \gamma$}

In the last section, we have:

\begin{equation}
\frac{w_j d_j \sum_j \norm{w_j}_2 d_j }{\norm{w_j}_2} = \sum_i \alpha_i y_i x_{ji}
\end{equation}

Therefore, when the w is not 0, we can take the l2-norm of the this two vectors
and get:

\begin{equation}
\norm{\frac{w_j d_j \sum_j \norm{w_j}_2 d_j }{\norm{w_j}_2}}_2
= \norm{\sum_i \alpha_i y_i x_{ji}}_2
\end{equation}

Because w is a vector, we know that:

\begin{equation}
\norm{\frac{w_j}{\norm{w_j}_2}}_2 = 1
\end{equation}

So we have:

\begin{equation}
\norm{w_j}_2 d_j \gamma = w_j^T \sum_i \alpha_i y_i x_{ji}
\end{equation}

\textbf{2.2.1.(c) Show that}

For the first point, from the previous problem, we would know that:

\begin{equation}
\norm{\sum_i \alpha_i y_i x_{ji}}_2 <= d_j \gamma
\end{equation}

And we have also proven that, when w is not zero:

\begin{equation}
\norm{\sum_i \alpha_i y_i x_{ji}}_2 = d_j \gamma
\end{equation}

Then with the $w_j = 0$, we should have:

\begin{equation}
\norm{\sum_i \alpha_i y_i x_{ji}}_2 < d_j \gamma
\end{equation}

For the second point, we would know that:

\begin{equation}
\frac{w_j d_j \sum_j \norm{w_j}_2 d_j }{\norm{w_j}_2} = \sum_i \alpha_i y_i x_{ji}
\end{equation}

\begin{equation}
w_j = \frac{\norm{w_j}_2}{d_j \gamma} \sum_i \alpha_i y_i x_ji
\end{equation}

We could know that $\norm{w_j}_2 > 0, d_j>0, \gamma>0$, therefore, we could
always have a $\eta_i$:

\begin{equation}
w_j = \eta_j \sum_i \alpha_i y_i x_{ji}, \eta_j > 0
\end{equation}

When $w_j = 0$, we could set $\eta_j = 0$. Therefore, we always have:

\begin{equation}
\eta_j >= 0
\end{equation}

\textbf{2.2.2. Encourage Sparsity}
From the 2.2.1.(c), we would know that:

\begin{equation}
if \norm{\sum_i \alpha_i y_i x_{ji}}_2 < d_j \gamma, \quad then \quad w_j = 0
\end{equation}

We can see that if the result of $y_i x_i \alpha_i$ is smaller than a certain
number, which is given by the regularization terms, then the parameter $w_j$
would tend to vanish to minimize the total loss.

\textbf{2.2.3. Kernelized version}
For the kernelized version, We could derive the graient for the minimal in the
same way:

\begin{equation}
\frac{\partial L}{\partial w_j} = \frac{1}{2} \frac{\partial (\sum_j d_j
\norm{w_i}_2)^2}{\partial w_j} - \frac{\partial \alpha_i y_i \sum_j w_j^T
\phi (x_{ji})}{\partial w_j}
\end{equation}

\begin{equation}
\frac{w_j d_j \sum_j \norm{w_j}_2 d_j }{\norm{w_j}_2} = \sum_i \alpha_i y_i \phi
(x_{ji)}
\end{equation}

\begin{equation}
w_j = \frac{\norm{w_j}_2}{d_j \gamma} \sum_i \alpha_i y_i \phi (x_ji)
\end{equation}

Similar with previous problem, we could have:

\begin{equation}
w_j = \eta_j \sum_i \alpha_i y_i \phi (x_{ji}), \eta_j > 0
\end{equation}

When $w_j = 0$, we could set $\eta_j = 0$. Therefore, we always have:

\begin{equation}
\eta_j >= 0
\end{equation}








\section{Collaboration}
I dicussed with Zheng Chen with problem 1.2 on how to calculate the parameters,
and he also informed me about 2.2.1(b).

\end{document}
